\documentclass[12pt, oneside]{article}
\usepackage{amssymb,amsmath}
\usepackage[margin=1in]{geometry}
\usepackage{textpos}
\usepackage{float}
%\usepackage{color}
\usepackage{graphicx}
\usepackage{tikz}
\usetikzlibrary{positioning}
\usepackage{tikz-3dplot}
\usepackage{pgfopts}
\usepackage{circuitikz}
\usepackage{afterpage}
\usepackage{structuralanalysis}

\begin{document}
\begin{textblock*}{4cm}(-1.7cm,-2.3cm)
\noindent {\scriptsize AE731 Exit Exam Summer 2019}
\end{textblock*}

%Do not modify other than putting your name where stated
\begin{textblock*}{8cm}(12.5cm,-1cm)
\noindent {Name: }
\end{textblock*}
%Do not modify other than typing your acknowledgement where stated
\begin{textblock*}{13.5cm}(-1.7cm,-1.8cm)
%\noindent \textit{\footnotesize Acknowledgement: Your acknowledgement for collaboration and other sources goes here. }
\end{textblock*}

\vspace{1cm}

%Do not modify other than typing the homework number after #
\begin{center}
\textbf{\Large AE 731 - Elasticity}
\end{center}


%Rest should contain your solution for the homework. Feel free to improvise in ways that you believe make grading easier.
\begin{enumerate}
%TODO change numbers

\item
Digital image correlation (DIC) can be used in stead of strain gages to measure strain during a tensile test.
Often, strain gages are attached to the back side of a specimen to confirm the DIC measurements.
After allowing another student to conduct a test for you, you realize they attached three separate gages at unknown angles instead of using a standard rosette.
Assuming the DIC measurements are correct, find  the angles for each of the strain gages.
\textbf{Note:} This drawing is for illustration only, do not assume anything about the sign or magnitude of the angles from this drawing.

\begin{figure}[H]
	\centering
	\begin{circuitikz}[scale=2]
		\draw[->,thick] (0,0) -- (4,0) node[right] {$x$};
		\draw[->,thick] (0,0) -- (0,2) node[above] {$y$};
		\draw (1,0.0) to[R=$\varepsilon_a$] (2.9319,0.5176);
		\draw (1,0.0) to[R=$\varepsilon_b$] (2.414,1.414);
		\draw (1,0.0) to[R=$\varepsilon_c$] (0.482,1.932);
		\begin{scope}[shift={(1,0.0)}]
			\draw (1.7,0) arc (0:15:1.7) node at (2.1,0.25) {$\theta_1$};
			\draw (1.545,0.414) arc (15:45:1.6) node at (1.5,0.9) {$\theta_2$};
			\draw (-.388,1.449) arc (105:45:1.5) node at (0.5,1.7) {$\theta_3$};
		\end{scope}
	\end{circuitikz}
\end{figure}
DIC measurements:
\begin{align*}
\begin{Bmatrix}
\epsilon_{xx} \\ \epsilon_{yy} \\ \epsilon_{xy}
\end{Bmatrix} =
\begin{Bmatrix}
0.006\\-0.002\\0.000
\end{Bmatrix}
\end{align*}
Strain gage measurements:
\begin{align*}
\begin{Bmatrix}
\varepsilon_{a} \\ \varepsilon_{b} \\ \varepsilon_{c}
\end{Bmatrix} =
\begin{Bmatrix}
0.00482\\0.00303\\-0.00095
\end{Bmatrix}
\end{align*}

\newpage
\afterpage{\null\newpage}

\item
Under what circumstances (if any) is the following symmetric 2D stress field in static equilibrium?
Assume all body forces (including inertia) are zero.
\begin{align}
	\sigma_{11} &= k_1 x_1 + 3 x_2^2\\
	\sigma_{22} &= 2x_1 + 4x_2\\
	\sigma_{12} = \sigma_{21} &= a + bx_1 + cx_1^2 + dx_2 + ex_2^2 + fx_1x_2
\end{align}

\newpage

\item
Determine the state of stress in a large plate, with a small hole, uniformly loaded in the y-direction remote from the hole.
Use the provided table (last page) to select the appropriate stress function.

\begin{figure}[H]
	\centering
	\includegraphics[width=0.3\textwidth]{./hole-remotey}
	\caption{Hole in an infinite sheet remotely loaded along y¬-direction.}
\end{figure}
\end{enumerate}

\afterpage{\null\newpage}
\newpage


\begin{figure}[H]
	\centering
	\includegraphics[width=0.9\textwidth]{./stress-functions}
	\caption{Stress functions}
\end{figure}
\end{document}
